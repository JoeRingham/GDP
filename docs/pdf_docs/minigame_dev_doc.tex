\documentclass[a4paper,12pt]{article}
%%%%%%%%%%%%%%%%%%%%%%%%%%%%%%%%%%%%%%%%%
% Contract
% Structural Definitions File
% Version 1.0 (December 8 2014)
%
% Created by:
% Vel (vel@latextemplates.com)
% 
% This file has been downloaded from:
% http://www.LaTeXTemplates.com
%
% License:
% CC BY-NC-SA 3.0 (http://creativecommons.org/licenses/by-nc-sa/3.0/)
%
%%%%%%%%%%%%%%%%%%%%%%%%%%%%%%%%%%%%%%%%%

%----------------------------------------------------------------------------------------
%	PARAGRAPH SPACING SPECIFICATIONS
%----------------------------------------------------------------------------------------

\setlength{\parindent}{0mm} % Don't indent paragraphs

\setlength{\parskip}{2.5mm} % Whitespace between paragraphs

%----------------------------------------------------------------------------------------
%	PAGE LAYOUT SPECIFICATIONS
%----------------------------------------------------------------------------------------

\usepackage{geometry} % Required to modify the page layout

\setlength{\textwidth}{16cm} % Width of the text on the page
\setlength{\textheight}{24.5cm} % Height of the text on the page

\setlength{\oddsidemargin}{0cm} % Width of the margin - negative to move text left, positive to move it right

% Uncomment for offset margins if the 'twoside' document class option is used
%\setlength{\evensidemargin}{-0.75cm} 
%\setlength{\oddsidemargin}{0.75cm}

\setlength{\topmargin}{-1.25cm} % Reduce the top margin

%----------------------------------------------------------------------------------------
%	FONT SPECIFICATIONS
%----------------------------------------------------------------------------------------

% If you are running Apple OS X, uncomment the next 4 lines and comment/delete the block below, you will now need to compile with XeLaTeX but your document will look much better

%\usepackage[cm-default]{fontspec}
%\usepackage{xunicode}

%\setsansfont[Mapping=tex-text,Scale=1.1]{Gill Sans}
%\setmainfont[Mapping=tex-text,Scale=1.0]{Hoefler Text}

%-------------------------------------------

\usepackage[utf8]{inputenc} % Required for including letters with accents
\usepackage[T1]{fontenc} % Use 8-bit encoding that has 256 glyphs

\usepackage{avant} % Use the Avantgarde font for headings
\usepackage{mathptmx} % Use the Adobe Times Roman as the default text font together with math symbols from the Sym­bol, Chancery and Com­puter Modern fonts

%----------------------------------------------------------------------------------------
%	SECTION TITLE SPECIFICATIONS
%----------------------------------------------------------------------------------------

\usepackage{titlesec} % Required for modifying section titles

\titleformat{\section} % Customize the \section{} section title
{\sffamily\large\bfseries} % Title font customizations
{\thesection} % Section number
{16pt} % Whitespace between the number and title
{\large} % Title font size
\titlespacing*{\section}{0mm}{7mm}{0mm} % Left, top and bottom spacing around the title

\titleformat{\subsection} % Customize the \subsection{} section title
{\sffamily\normalsize\bfseries} % Title font customizations
{\thesubsection} % Subsection number
{16pt} % Whitespace between the number and title
{\normalsize} % Title font size
\titlespacing*{\subsection}{0mm}{5mm}{0mm} % Left, top and bottom spacing around the title % Input the structure.tex file which specifies the document layout and style

%----------------------------------------------------------------------------------------

\begin{document}

\begin{center}
{\Huge How to use the Game Developer API}
\end{center}


%----------------------------------------------------------------------------------------
%	OVERVIEW SECTION
%----------------------------------------------------------------------------------------
\section{Overview}


The Game API is intended to have as little intrusion into the game making process as possible.
Games can be made with creation program or by hand, using any assortment of libraries.

The API is provided as a set of Javascript functions that are to be called on specific actions.
Exactly how these functions are called is left to the discretion on the implementor, but the game
must promise to execute them in given scenarios.


%----------------------------------------------------------------------------------------
%	THE GAME API SECTION
%----------------------------------------------------------------------------------------
\section{The Game API}

The API has 7 functions that can be called to make an impact on the user outside of the minigame.
The user has backpack which contains items they can use to modify their health, irrespective of any game mechanics
the minigame may have. This is NOT to say extra items inside the game cannot also affect the health and status values.

It is upto the game developer to also visualise the backpack; there is no standardised menu to use. The contents of the bag
are passed in when the game is launched, and then are updated when an item is used. Beyond this, the backpack can be displayed
in any way that is seen fit by the developer.

Additionally, when modifying health or status values, the new value should be read from the callback function.
The returned value may not always be exactly what was specified (e.g. if old health was 100 and we take away 
20, the value returned may not be 80). For this reason, you should not manually keep track of health. 
When a users health reaches 0, the game is over and the user is returned to the hub.
\\
The 7 functions are as follows:

\subsection*{finishGame(score, currency)}
This should be called when the game ends for any reason. If you wish to display a score screen for the game then
this should appear before this function is called. It will unload the game from memory and return to the hub area.\\
Parameters:
\begin{itemize}
	\setlength\itemsep{0em}
	\item score: the score of the game
	\item currency: the currency earned from playing the game
\end{itemize}

\newpage

\subsection*{useCarriable(carriableId, cb)}
A carriable is an item the user can use to restore health and status values. This function should be called when the
user wishes to utilise an item, which are all provided for you in the 'bag' object when the game is run.\\
Parameters:
\begin{itemize}
	\setlength\itemsep{0em}
	\item carriableId: the id of the carriable to use
	\item cb: the callback function, explained below
\end{itemize}

The callback function accepts the parameters (bag, health, statuses, avatarImage, symptoms) which are:
\begin{itemize}
	\setlength\itemsep{0em}
	\item bag: the updated contents of the players bag
	\item health: the new health value
	\item statuses: the new status values
	\item avatarImage: the new avatar image
	\item symptoms: the updated list of symptoms
\end{itemize}

\subsection*{getCarriableInfo(carriableId, cb)}
Gets the information for a carriableId, which will be needed to display the information and sprite to the user
so they can select it.\\
Parameters:
\begin{itemize}
	\setlength\itemsep{0em}
	\item carriableId: the ID of the carriable to query for information
	\item cb: the callback function, explained below
\end{itemize}

The callback function accepts the parameters (carriable) which are:
\begin{itemize}
	\setlength\itemsep{0em}
	\item carriable: the carriable configuration object
\end{itemize}

\subsection*{modifyHealth(changeVal, cb)}
Modifies the health by a \textbf{relative} amount. It is important to not the value specified may be modified by the server.
It is therefore important to use the value specified in the callback as the new health, and not calculate it yourself.\\
Parameters:
\begin{itemize}
	\setlength\itemsep{0em}
	\item changeVal: the amount to add/subtract from the health value
	\item cb: the callback function, explained below
\end{itemize}

The callback function accepts the parameters (health, avatar, symptoms) which are:
\begin{itemize}
	\setlength\itemsep{0em}
	\item health: the new health value
	\item avatar: the new avatar image
	\item symptoms: the new list of symptoms
\end{itemize}

\subsection*{modifyStatus(statusId, changeVal, cb)}
Modifies the value of a given status.\\
Parameters:
\begin{itemize}
	\setlength\itemsep{0em}
	\item statusId: the ID of the status to affect
	\item changeVal: the amount to add/subtract from the status value
	\item cb: the callback function, explained below
\end{itemize}

The callback function accepts the parameters (id, value) which are:
\begin{itemize}
	\setlength\itemsep{0em}
	\item id: the id of the specified status (the same as the one supplied)
	\item value: the new value for the status
\end{itemize}

\subsection*{getAvatarImage()}
Returns the avatar image that is current in use. It is in the form of an image object, not the raw Base64 encoding.

\subsection*{getAssetURL(asset)}
This is how extra assets are loaded. Returns the URL of where the asset is located. This must be manipulated into a tag
and then added to the canvas. e.g. images must be put in an <img> object, and Javascript files must be put in
a <script> tag.\\
Parameters:
\begin{itemize}
	\setlength\itemsep{0em}
	\item asset: a string specifying the asset to load relative to the entryObject
\end{itemize}	

%----------------------------------------------------------------------------------------
%	HOW TO STRUCTURE YOUR CODE SECTION
%----------------------------------------------------------------------------------------
\section{How to structure your code}
There must be exactly one entry script, which will be the script the framework calls. This entry script
can then, in turn, call other scripts and assets as desired. The entry script must contain a function "run"
as details below:

\subsection*{function run(api, canvas, assetBaseURL, startHp, statuses, bag)}
\begin{itemize}
	\setlength\itemsep{0em}
	\item a: 			the Game API, this is the object that will contain the callable functions as detailed earlier
	\item can : 			the HTML5 canvas. This is where the display will be rendered.
	\item assetBaseURL:	the base URL for the assets. It is the root folder of the game scripts.
	\item startHp:		the value of health the player currently has
	\item statuses:		all the different statuses the player has (e.g. bloodsugar for diabetes)
	\item bag:			the bag with the players carriable items
\end{itemize}

%----------------------------------------------------------------------------------------
%	SETTING UP THE DEVELOPMENT SCRIPT SECTION
%----------------------------------------------------------------------------------------
\section{Setting up the development script}

There is a development script available, which will allow you to simulate the server on a local machine. The responses are of course limited, but
it will allow you to see the functionality work.

The easiest way to set up local testing is with a few files, and be sure to include all other assets within an 'assets' folder:

\subsection*{example.html}
The following 4 lines should be written to \textit{example.html}. api.js is the local api testing file, and dummyGame.js is the script containing the \textit{run} function

\begin{verbatim}
<canvas id="canvas"></canvas>
<script src="api.js"></script>
<script src="dummyGame.js"></script>
<script src="example.js"></script>
\end{verbatim}

\subsection*{example.js}
This file will set up the canvas environment and launch the game

\begin{verbatim}
var c = document.getElementById("canvas");
var g = new GameLauncher();
g.launchGame(dummyGame, c);
\end{verbatim}

\section{Conclusion}
With this, you should now be able to create games using the framework, test it locally and be in a ready state for it to be uploaded and used by players!
\end{document}